\documentclass{article}
\usepackage[T2A]{fontenc}   % Кодировка шрифтов (T2A для кириллицы)
\usepackage[utf8]{inputenc} % Кодировка ввода (utf8)
\usepackage[russian]{babel}
\usepackage{tabularx}
\usepackage{amsmath}
\usepackage{amsfonts}
\usepackage{bm}
\usepackage{graphicx}%Proper images inserting
\usepackage{comment}
\usepackage{authblk}
\usepackage{makecell}
\renewcommand*{\Authand}{, } % Remove "and" between authors
\sloppy %To remove overfull\hbox messages from logs
\usepackage{csquotes}
\usepackage[backend=biber, style=ieee, url=false]{biblatex}
\addbibresource{My Library.bib}

\title{Метод тестирования исправности волоконных зеркал Фарадея}
\author[1]{Игорь Бучилко}
\author[1]{Леонид Лиокумович}
\affil[1]{Санкт-Петербургский политехнический Университет Петра Великого}

\date{Февраль 2025}

\begin{document}	
\maketitle
\begin{abstract}
В данной работе представлен новый метод тестирования волоконных зеркал Фарадея (FFRM), позволяющий точно определять угол фарадеевского вращения в условиях произвольного преобразования поляризации в подводящем тракте. Проведен анализ существующих методов тестирования FFRM, выявлены их ключевые недостатки, связанные с зависимостью от анизотропии подводящего тракта. Предложен модернизированный метод, основанный на использовании вращающейся полуволновой пластинки, который устраняет влияние анизотропии и обеспечивает более точное измерение угла фарадеевского вращения. Теоретическое обоснование метода подтверждено экспериментально: проведены измерения отклонения угла фарадеевского вращения для набора FFRM, показавшие высокую точность и воспроизводимость результатов. Предложенный метод может быть использован для тестирования FFRM в волоконно-оптических системах, включая датчики тока, интерферометры и системы квантовой криптографии. 
\end{abstract}

\section{Введение}
Зеркала Фарадея (далее FRM от англ. Faraday Rotator Mirror), разработанные в 1980-х годах,  стали широко применяться только последние 10 – 15 лет.  В общем случае FRM представляет из себя комбинацию невзаимного ротатора Фарадея и плоского зеркала\autocite{paschottaFaradayMirrors2022}. В исправном FRM угол фарадеевского вращения ротатора составляет 45°, в этом случае отраженное излучение сохраняет свою эллиптичность, а его азимут поворачивается на 90°. Волоконные зеркала Фарадея (далее FFRM, от англ. Fiber Faraday Rotator Mirror) дополнительно содержат подводящий оптический тракт, который обычно представляет собой отрезок SM-волокна длиной порядка нескольких метров. Примечательно, что эффект поворота азимута на 90° сохраняется и в FFRM, несмотря на наличие подводящего тракта с произвольной фазовой анизотропией.

Однако в реальных устройствах угол фарадеевского вращения может отклоняться от номинального значения из-за производственных дефектов или условий эксплуатации. В частности, в некоторых оптических схемах, FFRM могут подвергаться механическим нагрузкам, воздействию сильных магнитных полей или высоких температур. Это может привести к значительному отклонению угла фарадеевского вращения даже у изначально исправного зеркала.

Эффекты, связанные с отклонением угла фарадеевского вращения особенно важно учитывать при разработке волоконных датчиков тока \autocite{karabulutEffectFaradayMirror2019}, \autocite{wuInfluenceImperfectFaraday2022}, систем квантовой криптографии \autocite{wangEffectImperfectFaraday2013}, \autocite{sunPassiveFaradaymirrorAttack2011},\autocite{liangSecurityAnalysisContinuousvariable2019} и интерферометров на основе схемы Майкельсона.

Для отдельно взятого ротатора Фарадея оценка угла вращения может быть относительно просто выполнена с использованием измерительной схемы на основе двух поляризаторов \autocite{yinFaradayAngleAccuracy2022}. Однако наличие в составе FFRM подводящего тракта с произвольным преобразования поляризации значительно усложняет такую оценку.

Несмотря на важность данной задачи для практических приложений, в известной авторам литературе она освещена недостаточно подробно. В ряде работ \autocite{wanTwoinoneFaradayRotator2014}, \autocite{sunAllFiberOpticalFaraday2010},  \autocite{sunCompactAllfiberOptical2010} предложены схемы тестирования на основе поляризаторов, однако отсутствует обоснование применяемых методик.

В данной работе проводится теоретический анализ существующих подходов к оценке угла вращения ротатора в волоконном зеркале Фарадея, выявляются их потенциальные ограничения и источники ошибок. Кроме того, предлагается, теоретически обосновывается и экспериментально проверяется новый метод, позволяющий оценить реальный угол фарадеевского вращения в FFRM с помощью контролируемого изменения поляризации на входе тестируемого устройства. Предложенный метод лишён недостатков существующих подходов и обеспечивает более достоверную оценку угла вращения.

\section{Анализ известных подходов}
\begin{figure}[b]
	\centering
	\includegraphics[width=1\linewidth]{figures/existed_setup.png}
	\caption{Схема тестирования c фиксированным поляризатором}
	\label{fig:existed_setup}
\end{figure}

В исправном FRM угол фарадеевского вращения ротатора составляет 45°, в этом случае отраженное излучение сохраняет свою эллиптичность, а его азимут поворачивается на 90°.  Примечательно, что эффект поворота азимута на 90° сохраняется и при наличии подводящего тракта с произвольной фазовой анизотропией, то есть реализуется также и для FFRM. Это справедливо при условии, что степень поляризации излучения в тракте не снижается, то есть не происходит деполяризации. На практике этого довольно легко добиться использованием относительно коротких волоконных трактов и лазерных источников света. На этом ключевом свойстве зеркал Фарадея основан естественный потенциальный подход к тестированию FFRM. Таким образом, если на вход идеального FFRM подаётся линейно поляризованное излучение, то отражённое излучение также будет линейно поляризованным, но с ортогональной ориентацией. В случае, если угол Фарадеевского вращения отличается от номинальных 45°, указанное свойство FFRM выполняться не должно и состояние поляризации отраженного излучения не будет линейным и ортогональным входному. 

На рис.\ref{fig:existed_setup} представлена волоконно-оптическая схема, использующая такой подход к измерению. Она включает в себя источник излучения (OS), измеритель оптической мощности (OPM), Y-разветвитель, волоконный поляризатор (P) и тестируемое FFRM с одномодовым волокно (SMF). От источника не требуется высокая когерентность, однако его спектр должен соответствовать рабочему диапазону FFRM, т.е. как правило это должен быть лазерный источник, хотя не обязательно высокого уровня когерентности. В части схемы слева от поляризатора допускается использование SMF компонентов. Состояние поляризации излучения на выходе источника также может быть произвольным. Ключевое требование — стабильность поляризации излучения на входе в поляризатор и, соответственно, стабильность оптической мощности $P_{in}$ на его выходе. Хотя использование PM-компонентов и линейно-поляризованного лазера позволяет упростить настройку и повысить стабильность $P_{in}$, строго говоря, это не является необходимым условием для выполнения измерений. Важно, что со стороны подключения FFRM следует использовать волокно с сохранением поляризации (PMF), при этом необходимо согласовать ориентацию его оптической оси с осью поляризатора.

Следует отметить, что аналогичная схема уже использовалась в работе \autocite{wanTwoinoneFaradayRotator2014}, однако ее применение не было теоретически обосновано.

В предлагаемой схеме на вход FFRM поступает линейно поляризованное излучение через PM-волокно. Согласно указанному выше принципу преобразования поляризации при отражении от исправного FFRM, отражённое излучение будет иметь ортогональную ориентацию и не сможет пройти через поляризатор, в результате чего мощность $P_{out}$ после поляризатора, а также регистрируемая мощность $P_{r}$ на OPM, должны быть близки к нулю. Если же зеркало неисправно и угол поворота поляризации отличается от 45°, то отраженное излучение частично пройдёт через поляризатор, и $P_{in}$ станет отличной от нуля.

Таким образом, описанный подход позволяет качественно оценить исправность FFRM. Однако для перехода к количественным измерениям и корректной интерпретации результатов требуется строгое определение связи между регистрируемой мощностью $P_{r}$ и фактическим значением угла Фарадеевского вращения в FFRM.  


Рассмотрим работу этой схемы  подробнее, используя формализм матриц и векторов Джонса.
Пусть $E_{in}$ - вектор Джонса в точке $in$, то есть сразу после поляризатора. В этом случае он может быть записан в следующей форме:
\begin{equation}
	\bm{E_{in}} = \begin{pmatrix} 1 \\ 0 \end{pmatrix}\sqrt{P_{in}} \\
\end{equation}

Матрицы Джонса $SM$ и $PM$ для SM и PM волокна в соответствии с [...] могут быть записаны в следующем виде:
\begin{equation}
	SM = 
	\begin{pmatrix}
		A & B \\
		-B^* & A^*
	\end{pmatrix},\qquad 
	PM = 
	\begin{pmatrix}
		e^{i\frac{\phi}{2}} & 0 \\
		0 & e^{-i\frac{\phi}{2}}
	\end{pmatrix},	
\end{equation}
Здесь $\phi$ – разность фаз между собственными модами PM волокна, а условие $|A|^2 + |B|^2 = 1$ отражает отсутствие потерь в SM волокне

Матрицы Джонса для плоского зеркала $M$, поляризатора  $P$ и невзаимного ротатора $R_\theta$ могут быть записаны соответственно в следующем виде:
\begin{equation}
	M = 
	\begin{pmatrix}
		1 & 0 \\
		0 & 1
	\end{pmatrix},\qquad 		
	P = 
	\begin{pmatrix}
		1 & 0 \\
		0 & 0
	\end{pmatrix},\qquad 
	R_\theta = 
	\begin{pmatrix}
		\cos\theta & -\sin\theta \\
		\sin\theta & \cos\theta
	\end{pmatrix},		
\end{equation}
Конкретный вид матриц, а также их трансформация при распространении света в обратном направлении определяется преобразованием координатного базиса при отражении. Мы в расчетах используем базис, не привязанный к направлению распространения света, соответственно матрицы имеют вид представленный выше. Матрицы взаимных элементов при прохождении излучения в обратном направлении транспонируются, а матрица невзаимного ротатора Фарадея сохраняет свой вид.


С учетом вышесказанного вектор Джонса $\bm{E_{out}}$ для излучения в точке $out$ будет определяться следующим образом:
\begin{equation}
	\bm{E_{out}} =P \cdot (SM\cdot PM)^T\cdot (R_\theta \cdot M \cdot R_\theta)\cdot (SM\cdot PM)\cdot \bm{E_{in}}
\end{equation}	

Подставляя в эту формулу явные выражения для матриц и векторов Джонса введенные ранее, получаем:
\begin{equation}
	\bm{E_{out}} =  e^{i\phi} \sqrt{P_{in}} \begin{pmatrix} A^2 + {B^*}^2 \\ 0 \end{pmatrix}\cos 2 \theta
\end{equation}

В соответствии с \autocite{azzamEllipsometryPolarizedLight1977} мощность $P_{out}$ с точностью до константы может быть определена как: 	
\begin{equation}
	P_{out} = \bm{E_{out}^+} \cdot \bm{E_{out}} = \left|\bm{E_{out}} \right|^2 
\end{equation}

В результате получаем следующее выражение для мощности излучения в точке $out$:
\begin{equation}
	P_{out} = P_{in} \cdot \left| A^2 + {B^*}^2 \right|^2 \cdot \cos^2 2 \theta
\end{equation}

Окончательное выражение для мощности $P_r$ регистрируемой на OPM может быть записано в следующем виде:
\begin{equation}
	\label{eq:Iout_1_1}
	P_{r} = P_0 \cdot \left| A^2 + {B^*}^2 \right|^2 \cdot \cos^2 2 \theta
\end{equation}

Здесь $P_0$ – опорная мощность учитывающая, мощность излучения источника, согласование входного излучения с осью пропускания поляризатора, а также неучтенное ранее затухание в элементах схемы. При этом величину $P_0$ можно достаточно точно определить экспериментально, измерив регистрируемую мощность при замене FFRM на обычное плоское зеркало. Из выражения (\ref{eq:Iout_1_1}) видно, что $P_{r}$ зависит как от угла фарадеевского вращения ротатора $\theta$, так и от параметров подводящего тракта, задаваемых матрицей $SM$. Неопределенность, связанная с этими параметрами, делает схему, в общем случае, непригодной для точного определения угла $\theta$.  Особенно ярко это проявляется, когда  $\left| A^2 + {B^*}^2 \right|=0$. В этом случае результат измерений не зависит от $\theta$ и неотличим от результата для идеального FFRM. Это делает извлечение информации об угле вращения ротатора FRM полностью невозможным. Интересно заметить, что в этом случае подводящей тракт ведет себя эквивалентно четвертьволновой пластинке. 


Однако данная схема все же может работать корректно и использоваться для оценки  угла фарадеевского вращения ротатора FFRM. Это возможно, если подводящий тракт FFRM выполнен из линейного PM-волокна, а его быстрая ось согласована с осью пропускания поляризатора.  В этом случае $\left| A^2 + {B^*}^2 \right| = 1$, и выходная интенсивность определяется как: 
\begin{equation}
	\label{eq:Iout_1_2}
	P_{r} =  P_0 \cos^2 2 \theta 
\end{equation}

Из выражения (\ref{eq:Iout_1_2}) видно, что $P_{or}$ определяется только углом  $\theta$ и мощностью $P_0$, и не зависит от преобразования поляризации в подводящем тракте. При этом величину $P_0$ можно достаточно точно определить экспериментально, измерив выходную интенсивность при замене FFRM на обычное плоское зеркало. Таким образом, измерение выходной мощности $P_{r}$ позволяет корректно определить угол фарадеевского вращения ротатора. В качестве альтернативы линейному PM-волокну можно использовать волокно с круговым двулучепреломлением (SPUN), что устраняет необходимость согласования с поляризатором. 


\section{Предлагаемый метод}
\begin{figure}[b]
	\centering
	\includegraphics[width=1\linewidth]{figures/proposed_setup.pdf}
	\caption{Модернизированная схема тестирования FRM}
	\label{fig:proposed_setup}
\end{figure} 
Основным недостатком рассмотренных подходов является их чувствительность к преобразованию поляризации в подводящем тракте FFRM.  Если подводящий тракт преобразует входное излучение в циркулярно поляризованное, то  выходная поляризация будет соответствовать поляризации на выходе идеального FFRM, независимо от фактического угла вращения ротатора. В общем случае при фиксированной поляризации входного излучения, FFRM с углом вращения ротатора $\theta$ может проявлять себя как FFRM с эффективным углом вращения от $\theta$ до $45^\circ$ в зависимости от анизотропии подводящего тракта \autocite{buchilkoAnalysisStatePolarization2024}.

В случае тестирования FFRM с подводящим трактом на основе PM-волокна можно использовать модифицированные подходы, предложенные в предыдущем разделе. Однако если подводящий тракт осуществляет произвольное преобразование поляризации, например, при использовании SM-волокна, измерения с фиксированной входной поляризацией принципиально не позволяют определить угол вращения ротатора FFRM. В таких условиях возможным решением может быть контролируемое изменение состояния поляризации на входе FFRM.

Модернизируем схему изображенную на рис.\ref{fig:existed_setup}, добавив  между поляризатором и подводящим трактом полуволновую пластинку повернутую под углом $\alpha$ (рис.\ref{fig:proposed_setup}). Для сохранения линейной поляризации между поляризатором и полуволновой пластинкой используется PM волокно. Пусть как и ранее $\theta$ и $I_0$ -  угол вращения ротатора FFRM и некоторая постоянная опорная интенсивность.  Уравнение для выходной интенсивности $I_{out}$ в этом случае  может быть записано как:
\begin{equation}
	\label{eq:Iout_our}
	I_{out} =  I_0 \left( 1 - 4 Z^2 \cdot \sin^2(4\alpha - \alpha_0)\right)  \cos^2 2 \theta
\end{equation}

где
\begin{equation}
	Z = \sqrt{Im^2\left[\frac{A^2-B^2}{2}\right]+Im^2[AB]}, \qquad  \tg\alpha_0 = \frac{2Im[AB]}{Im\left[A^2 - B^2\right]}
\end{equation}

Рассмотрим (\ref{eq:Iout_our}) подробнее. Видно, что существует угол $\alpha = \frac{\alpha_0}{4}$, при котором выходная интенсивность достигает своего максимального значения
\begin{equation}
	\label{eq:Imax}
	I_{max} = I_0\cos^22\theta
\end{equation}

Иными словами, для любого преобразования поляризации подводящим трактом можно подобрать такой угол поворота полуволновой пластинки, при котором выходная интенсивность будет достигать своего максимального значения, которое не зависит от анизотропии подводящего тракта и определяется углом фарадеевского вращения ротатора FFRM. 
В соответствии с (\ref{eq:Imax}) для определения угла $\theta$ необходимо провести два измерения выходной интенсивности. Первое с обычным зеркалом ($\theta=0$), для определения $I_0$, а второе с тестируемым FFRM. Тогда, если $P_M$ и $P_{FFRM}$ – это соответственно интенсивности выходного излучения для обычного зеркала и FFRM, то угол вращения ротатора может быть вычислен по следующей формуле:
\begin{equation}
	\label{eq:theta}
	\theta=\frac{1}{2}\arccos\sqrt{\frac{P_{FFRM}}{P_M}}
\end{equation}

В обоих измерениях необходимо добиваться максимальной выходной интенсивности путем подбора соответствующего угла поворота полуволновой пластинки. 

\section{Экспериментальная проверка предложенного метода}
\begin{figure}[b]
	\centering
	\includegraphics[width=1\linewidth]{figures/experimental_setup.pdf}
	\caption{Практическая схема тестирования FRM}
	\label{fig:experimantalScheme}
\end{figure}

Для проверки предложенного метода была собрана измерительная схема, показанная на рис.\ref{fig:experimantalScheme}. В отличие от схемы, представленной на рис.\ref{fig:proposed_setup}, здесь отсутствует в явном виде поляризатор, однако его роль играет циркулятор, который пропускает только одну линейную поляризационную моду. 
С помощью этой схемы по формуле (\ref*{eq:theta}) были определены отклонения угла фарадеевского вращения от 45° для набора FRM. В качестве опорного зеркала для определения $P_M$ использовалось зеркало Thorlabs P5-SMF28ER-P01-1. Результаты измерений приведены в таблице \ref{tabular:results}.
\begin{table}[h]
	\caption{Результаты измерений неидеальности зеркал Фарадея}
	\label{tabular:results}
		\begin{tabularx}{\textwidth}{|X|c|c|}
			\hline
			\thead{Зеркало} & \thead{Интенсивность выходного \\ излучения, дБм} & \thead{Неидеальность FRM, град} \\
			\hline
			\makecell{Thorlabs \\ P5-SMF28ER-P01-1} & -11.28 & - \\	
			\hline
			\makecell{AFW Technologies \\ SN: 17049033} & -47.67 & 0.43 \\
			\hline
			\makecell{AFW Technologies \\ SN: 17049037}	& -50,70 & 0.31 \\
			\hline
			\makecell{AFW Technologies \\ SN: 17049031}	& -41.98 & 0.83 \\
			\hline
			\makecell{AFW Technologies \\ SN: 17049035}	& -53.06 & 0.23 \\
			\hline
			\makecell{AFW Technologies \\ SN: 17049029} & -42.54 & 0.78 \\
			\hline
		\end{tabularx}
\end{table}

В исправных FFRM отклонение угла фарадеевского вращения обычно составляет порядка 1° \autocite{paschottaFaradayMirrors2022}, тогда как у поврежденных FFRM оно может достигать десятков градусов, вплоть до 45° в том случае, если эффект Фарадея не проявляется. Соответственно из таблицы 1 видно, что все проверенные FFRM можно считать исправными. Следует, однако, отметить, что при измерении не учитывались конечная экстинкция циркулятора, а также различие в коэффициентах отражения обычного зеркала и FFRM. 

\section{Заключение}
В работе проанализированы представленные в научной литературе подходы к тестированию зеркал Фарадея и предложен улучшенный метод такого тестирования. Показано, что в силу произвольности преобразования поляризации вводным трактом зеркала Фарадея, подходы использующие измерительную схему с фиксированными параметрами могут потенциально привести к некорректным результатам. Показано, что для корректной оценки отклонения угла фарадеевского вращения в FRM достаточно использования вращающейся полуволновой пластинки как части подводящего тракта. Приведены результаты измерения неидеальности набора FRM по предложенной методике.

\appendix
\numberwithin{equation}{section}
\section{Доказательство выражения (\ref{eq:Iout_our})}
Рассмотрим схему, представленную на рис.\ref{fig:proposed_setup}.  Пусть $\alpha$ – угол поворота полуволновой пластинки, а $\phi$ – разность фаз между собственными модами PM волокна. Тогда матрицы Джонса для PM волокна и полуволновой пластинки могут быть записаны в следующем виде:
\begin{equation}
	PM = 
	\begin{pmatrix}
		e^{i\frac{\phi}{2}} & 0 \\
		0 & e^{-i\frac{\phi}{2}}
	\end{pmatrix},\qquad 
	HWP_\alpha = 
	\begin{pmatrix}
		\cos 2 \alpha & \sin 2 \alpha \\
		\sin 2 \alpha & -\cos 2 \alpha
	\end{pmatrix}
\end{equation}

Матрица Джонса описывающая оптическую схему правее поляризатора в отличие от (\ref{eq:M45}) будет иметь следующий вид:
\begin{equation}
	SM \cdot HWP_\alpha \cdot PM =
	\begin{pmatrix} 
		A_\alpha 	&& 	B_\alpha \\
		-B_\alpha^* && 	A_\alpha^*
	\end{pmatrix}
\end{equation}

где
\begin{equation}
	\label{eq:alpha_def}
	A_\alpha = \left( A\cos2\alpha + B\sin2\alpha\right) e^{i\frac{\phi}{2}},
	B_\alpha = \left( A\sin2\alpha - B\cos2\alpha\right) e^{-i\frac{\phi}{2}}
\end{equation}

Выражение для выходной интенсивности $I_{out}$ может быть получено из (\ref{eq:Iout_1_1}) заменой A и B на соответственно $A_\alpha$ и $B_\alpha$.

Рассмотрим отдельно множитель $\left| A^2 + {B^*}^2 \right|^2$ из выражения (\ref{eq:Iout_1_1}):
\begin{equation}
	\begin{split}
		\left| A^2 + {B^*}^2 \right|^2 &= |A|^4 + |B|^4 + 2Re\left[ (A B)^2\right]\\
		&=\left( |A|^2 + |B|^2\right)^2 - 2\left(|A B|^2 - Re\left[ (A B)^2\right]\right)\\
	\end{split}
\end{equation}	

Учитывая, что $|A|^2 + |B|^2 = 1$ и что $\forall Z \in \mathbb{C}$ выполняется равенство $Re\left[ Z^2\right] = Re^2\left[ Z\right] - Im^2\left[ Z\right]
$, приходим к следующему выражению

\begin{equation}
	\begin{split}
		\left| A^2 + {B^*}^2 \right|^2 	&=\left( |A|^2 + |B|^2\right)^2 - 4Im^2[A B] = 1 - 4Im^2[A B]
	\end{split}	
\end{equation}

Откуда получаем:
\begin{equation}
	\label{eq:Iout_our_1}
	I_{out} =  I_0 \left( 1 - 4Im^2[A_\alpha B_\alpha]\right)  \cos^2 2 \theta
\end{equation}

С учетом (\ref{eq:alpha_def}) $Im[A_\alpha B_\alpha]$ можно записать в следующем виде:
\begin{equation}
		Im[A_\alpha B_\alpha]  = Im\left[ \frac{A^2-B^2}{2} \right]\sin 4 \alpha - Im[AB]\cos 4 \alpha = Z \cdot \sin(4\alpha - \alpha_0)
\end{equation}


где
\begin{equation}
	Z = \sqrt{Im^2\left[\frac{A^2-B^2}{2}\right]+Im^2[AB]}, \qquad \tg\alpha_0 = \frac{2Im[AB]}{Im\left[A^2 - B^2\right]}
\end{equation}

Таким образом получаем окончательное выражение для интенсивности выходного излучения: 
 \begin{equation}
 	I_{out} =  I_0 \left( 1 - 4 Z^2 \cdot \sin^2(4\alpha - \alpha_0)\right)  \cos^2 2 \theta
 \end{equation}
 

\printbibliography
\end{document}
