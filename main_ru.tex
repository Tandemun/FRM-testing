\documentclass{article}
\usepackage[T2A]{fontenc}   % Кодировка шрифтов (T2A для кириллицы)
\usepackage[utf8]{inputenc} % Кодировка ввода (utf8)
\usepackage[russian]{babel}
\usepackage{tabularx}
\usepackage{amsmath}
\usepackage{amsfonts}
\usepackage{bm}
\usepackage{graphicx}%Proper images inserting
\usepackage{comment}
\usepackage{authblk}
\usepackage{makecell}
\usepackage{physics}
\renewcommand*{\Authand}{, } % Remove "and" between authors
\sloppy %To remove overfull\hbox messages from logs
\usepackage{csquotes}
\usepackage[backend=biber, style=ieee, url=false]{biblatex}
\addbibresource{My Library.bib}

\title{Метод тестирования исправности волоконных зеркал Фарадея}
\author[1]{Игорь Бучилко}
\author[1]{Леонид Лиокумович}
\affil[1]{Санкт-Петербургский политехнический Университет Петра Великого}

\date{Февраль 2025}

\begin{document}	
\maketitle
\begin{abstract}
В данной работе представлен новый метод тестирования волоконных зеркал Фарадея (FFRM), позволяющий точно определять угол фарадеевского вращения в условиях произвольного преобразования поляризации в подводящем тракте.
Проведен анализ существующих методов тестирования FFRM, выявлены их ключевые недостатки, связанные с зависимостью от анизотропии подводящего тракта.
Предложен модернизированный метод, основанный на использовании вращающейся полуволновой пластинки, который устраняет влияние анизотропии и обеспечивает более точное измерение угла фарадеевского вращения.
Теоретическое обоснование метода подтверждено экспериментально: проведены измерения отклонения угла фарадеевского вращения для набора FFRM, показавшие высокую точность и воспроизводимость результатов.
Предложенный метод может быть использован для тестирования FFRM в волоконно-оптических системах, включая датчики тока, интерферометры и системы квантовой криптографии.
\end{abstract}

\section{Введение}
Зеркала Фарадея (далее FRM от англ. Faraday Rotator Mirror), разработанные в 1980-х годах,  стали широко применяться только последние 10 – 15 лет.
В общем случае FRM представляет из себя комбинацию невзаимного ротатора Фарадея и плоского зеркала\autocite{paschottaFaradayMirrors2022}.
В исправном FRM угол фарадеевского вращения ротатора составляет 45°, в этом случае отраженное излучение сохраняет свою эллиптичность, а его азимут поворачивается на 90°.
Волоконные зеркала Фарадея (далее FFRM, от англ. Fiber Faraday Rotator Mirror) дополнительно содержат подводящий оптический тракт, который обычно представляет собой отрезок SM-волокна длиной порядка нескольких метров.
Примечательно, что эффект поворота азимута на 90° сохраняется и в FFRM, несмотря на наличие подводящего тракта с произвольной фазовой анизотропией.

Однако в реальных устройствах угол фарадеевского вращения может отклоняться от номинального значения из-за производственных дефектов или условий эксплуатации.
В частности, в некоторых оптических схемах, FFRM могут подвергаться механическим нагрузкам, воздействию сильных магнитных полей или высоких температур.
Это может привести к значительному отклонению угла фарадеевского вращения даже у изначально исправного зеркала.

Эффекты, связанные с отклонением угла фарадеевского вращения особенно важно учитывать при разработке волоконных датчиков тока \autocite{karabulutEffectFaradayMirror2019}, \autocite{wuInfluenceImperfectFaraday2022}, систем квантовой криптографии \autocite{wangEffectImperfectFaraday2013}, \autocite{sunPassiveFaradaymirrorAttack2011},\autocite{liangSecurityAnalysisContinuousvariable2019} и интерферометров на основе схемы Майкельсона.

Для отдельно взятого ротатора Фарадея оценка угла вращения может быть относительно просто выполнена с использованием измерительной схемы на основе двух поляризаторов \autocite{yinFaradayAngleAccuracy2022}.В случае ротатора с зеркалом для измерения угла надо разделить входящий и отраженный лучи, но принцип измерения по углу поворота азимута поляризации также относительно прост. Однако наличие в составе FFRM подводящего тракта с произвольным преобразованием поляризации значительно усложняет задачу измерения угла ротатора.

Несмотря на важность данной задачи для практических приложений, в известной авторам литературе она освещена недостаточно подробно.
В ряде работ \autocite{wanTwoinoneFaradayRotator2014}, \autocite{sunAllFiberOpticalFaraday2010},  \autocite{sunCompactAllfiberOptical2010} предложены схемы тестирования на основе поляризаторов, однако отсутствует обоснование применяемых методик.

В данной работе проводится теоретический анализ существующих подходов к оценке угла вращения ротатора в волоконном зеркале Фарадея, выявляются их потенциальные ограничения и источники ошибок.
Кроме того, предлагается, теоретически обосновывается и экспериментально проверяется новый метод, позволяющий оценить реальный угол фарадеевского вращения в FFRM с помощью контролируемого изменения поляризации на входе тестируемого устройства.
Предложенный метод лишён недостатков существующих подходов и обеспечивает более достоверную оценку угла вращения.

\section{Анализ известных подходов}
\begin{figure}[b]
	\centering
	\includegraphics[width=1\linewidth]{figures/existed_setup.png}
	\caption{Схема тестирования с фиксированным поляризатором}
	\label{fig:existed_setup}
\end{figure}

В исправном FRM угол фарадеевского вращения ротатора составляет 45°, тогда отраженное излучение сохраняет свою эллиптичность, а его азимут поворачивается на 90°.
Указанное свойство сохраняется и при наличии перед FRM подводящего тракта с произвольной фазовой анизотропией, то есть выполняется и для FFRM.
Следует отметить, что это справедливо при условии, что степень поляризации излучения в тракте не снижается, то есть не происходит деполяризации.
Но на практике обычно это условие выполняется, особенно для относительно коротких волоконных трактов и лазерных источников света.
На указанном ключевом свойстве зеркал Фарадея основан естественный потенциальный подход к тестированию FFRM.
Если на вход идеального FFRM подаётся линейно поляризованное излучение, то отражённое излучение от данного элемента также будет линейно поляризованным, но с ортогональной ориентацией.
В случае, если угол Фарадеевского вращения отличается от номинальных 45°, состояние поляризации отраженного излучения может отличаться от линейной поляризации, повернутой ортогонально входной.
Эта логика позволяет предложить подход к тестированию , на основе схемы, показанный на рис.\ref{fig:existed_setup}.
Схема включает в себя источник излучения (OS), измеритель оптической мощности (OPM), Y-разветвитель, волоконный поляризатор (P) и тестируемое FFRM с одномодовым волокном (SMF).
От источника не требуется высокая когерентность, однако его спектр должен соответствовать рабочему диапазону FFRM, т.е. как правило это должен быть лазерный источник, хотя не обязательно высокого уровня когерентности.
В части схемы слева от поляризатора допускается использование SMF компонентов.
Состояние поляризации излучения на выходе источника также может быть произвольным.
Ключевое требование — стабильность поляризации излучения на входе в поляризатор и, соответственно, стабильность оптической мощности $P_{in}$ на его выходе.
Хотя использование PM-компонентов и линейно-поляризованного лазера позволяет упростить настройку и повысить стабильность $P_{in}$, строго говоря, это не требуется для выполнения измерений.
Важно, что со стороны подключения FFRM следует использовать волокно с сохранением поляризации (PMF), при этом необходимо согласовать ориентацию его оптической оси с осью поляризатора.

Следует отметить, что схема такого типа использовалась в работе \autocite{wanTwoinoneFaradayRotator2014}, однако ее применение не было теоретически обосновано.

В схеме на рис.\ref{fig:existed_setup} линейно поляризованное излучение поступает на вход FFRM  через PM-волокно поляризатора.
Согласно указанному выше свойству преобразования поляризации при отражении от исправного FFRM, отражённое излучение будет иметь ортогональную ориентацию и будет подавлено поляризатором. В результате мощность $P_{out}$ после поляризатора, а также регистрируемая мощность $P_{r}$ на OPM, окажутся нулевыми.
Если же зеркало неисправно и состояние поляризации отраженного излучения будет отличаться от линейной и ортогональной оси поляризатора.
Тогда отраженное излучение частично пройдёт через поляризатор, и $P_{out}$ станет отличной от нуля измеритель мощности зафиксирует некоторый уровень мощности $P_{r}$.

Таким образом, описанный подход должен позволить обнаружить неисправность FFRM.
Однако обоснования такого метода измерения и корректной интерпретации результатов требуется определить связь между регистрируемой мощностью $P_{r}$ и фактическим значением угла Фарадеевского вращения в FFRM.

Рассмотрим работу этой схемы  подробнее, используя формализм матриц и векторов Джонса.
Пусть $\bm{E_{in}}$ - вектор Джонса на выходе волоконного поляризатора.
Он соответствует линейно поляризованному излучению и может быть записан в форме:
\begin{equation}
	\label{eq:Ein}
	\bm{E_{in}} = \begin{pmatrix} 1 \\ 0 \end{pmatrix}\sqrt{P_{in}} \\
\end{equation}

Где учтена обычно используемая связь вектора Джонса $\bm{E}$ и мощности излучения $P$, которая с точностью до константы может быть определена выражением \autocite{azzamEllipsometryPolarizedLight1977}: 	
\begin{equation}
	\label{eq:P-E}
	P = \bm{E^\dagger} \cdot \bm{E} = \left|\bm{E} \right|^2 
\end{equation}

Матрицы Джонса $\bm{PM}$ и $\bm{SM}$ для PM и SM волокна в соответствии с [...] могут быть записаны в следующем виде:
\begin{equation}
\label{eq:PM-SM}
    \bm{PM} = 
    \begin{pmatrix}
    e^{i\frac{\phi}{2}} & 0 \\
    0 & e^{-i\frac{\phi}{2}}
    \end{pmatrix},
    \qquad 
    \bm{SM} = 
    \begin{pmatrix}
    A & B \\
    -B^* & A^*
    \end{pmatrix},
\end{equation}
где $\phi$ – разность фаз между собственными модами PM волокна, а комплексные коэффициенты \textit{А} и \textit{B}, удовлетворяющие только условию $|A|^2 + |B|^2 = 1$, формируют матрицу произвольной фазовой анизотропии, описывающей произвольную анизотропную системы без поляризационно-зависимых потерь.
Именно такая матрица Джонса обычно используется для описания отрезков обычных SM-волокон с относительно малыми потерями.
Кроме указанного условия на сумму квадратов модулей, элементы \textit{A} и \textit{B} могут быть произвольны, что соответствует в общем случае произвольной непредсказуемой трансформации состояния поляризации света, прошедшего через отрезок SM-волокна к тому же зависящей от изменения положения волокна, его нагреву, натяжению и т.п..
Строго говоря приведенные матрицы могут также иметь множители, учитывающие поляризационно-независимые потери и общий сдвиг фазы излучения, но для данного рассмотрения они не требуются и их можно опустить.

Матрицы Джонса для плоского зеркала $M$, поляризатора  $P$ и невзаимного ротатора $R_\theta$ могут быть записаны соответственно в следующем виде:
\begin{equation}
	\label{eq:M-P-R}
	\bm{M} = 
	\begin{pmatrix}
		1 & 0 \\
		0 & 1
	\end{pmatrix},\qquad 		
	\bm{P} = 
	\begin{pmatrix}
		1 & 0 \\
		0 & 0
	\end{pmatrix},\qquad 
	\bm{R_\theta} = 
	\begin{pmatrix}
		\cos\theta & -\sin\theta \\
		\sin\theta & \cos\theta
	\end{pmatrix},		
\end{equation}
Конкретный вид матриц Джонса, а также их трансформация при распространении света в обратном направлении вообще говоря может быть разным в зависимости от выбранного варианта преобразования координатного базиса при отражении.
Мы в нашем анализе используем базис, не привязанный к направлению распространения света.
Соответственно матрица зеркала имеет вид представленный выше, матрицы взаимных элементов при прохождении излучения в обратном направлении транспонируются, а матрица невзаимного ротатора Фарадея сохраняет свой вид.

С учетом вышесказанного, для излучения, прошедшего через поляризатор в обратном направлении, вектор Джонса $\bm{E_{out}}$ будет определяться следующим произведением:
\begin{equation}
	\bm{E_{out}} = \bm{P} \cdot (\bm{SM}\cdot \bm{PM})^T\cdot (\bm{R_\theta} \cdot \bm{M} \cdot \bm{R_\theta})\cdot (\bm{SM}\cdot \bm{PM})\cdot \bm{E_{in}}
\end{equation}	
Подставляя в эту формулу явные выражения для матриц и векторов Джонса (\ref{eq:Ein}), (\ref{eq:PM-SM}) и (\ref{eq:M-P-R}), получим:
\begin{equation}
	\bm{E_{out}} =  e^{i\phi} \sqrt{P_{in}} \begin{pmatrix} A^2 + {B^*}^2 \\ 0 \end{pmatrix}\cos 2 \theta
\end{equation}

В результате с учетом (\ref{eq:P-E}) получим следующее выражение для мощности излучения $P_{out}$:
\begin{equation}
	P_{out} = P_{in} \cdot \left| A^2 + {B^*}^2 \right|^2 \cdot \cos^2 2 \theta
\end{equation}

Выражение для мощности $P_r$ регистрируемой на OPM может быть записано в следующем виде:
\begin{equation}
	\label{eq:Iout_1_1}
	P_{r} = P_0 \cdot \left| A^2 + {B^*}^2 \right|^2 \cdot \cos^2 2 \theta
\end{equation}

Здесь $P_0$ – опорное значение мощности учитывающее кроме значения $P_{in}$ еще поляризационно независимое затухание в элементах схемы, через которые проходит излучение на пути до OPM.
Определение $P_0$ требует отдельной калибровки, но пока будем полагать, что оно известно.
Из выражения (\ref{eq:Iout_1_1}) видно, что $P_{r}$ зависит как от угла фарадеевского вращения ротатора $\theta$, так и от параметров подводящего тракта, задаваемых коэффициентами $A$ и $B$ матрицы $SM$.
Нетрудно видеть, что  $0\le\left| A^2 + {B^*}^2 \right|\le1$.
Причем если $\left| A^2 + {B^*}^2 \right|=0$, то результат измерений такой же, как результат для идеального FFRM, независимо от фактического значения  $\theta$.
Поэтому неопределенность, связанная наличием отрезка SM-волокна перед FRM, делает рассмотренную схему, в общем случае, непригодной для определения угла $\theta$.

Нужно отметить, что данная схема может работать корректно и использоваться для оценки  угла $\theta$ , если подводящий тракт FFRM выполнен из PM-волокна.
Учтем, что при стандартной стыковке PM-волокон их оси согласованы, а также, что для РM-волокна согласно виду матрицы $PM$ коэффициент \textit{B} равен нулю, и $\left| A^2 + {B^*}^2 \right| = 1$.
Тогда получим, что: 
\begin{equation}
	\label{eq:Iout_1_2}
	P_{r} =  P_0 \cos^2 2 \theta 
\end{equation}

Из выражения (\ref{eq:Iout_1_2}) видно, что $P_{or}$ определяется только углом  $\theta$ и мощностью $P_0$, и не зависит от преобразования поляризации в подводящем PM-волокне.
При этом величину $P_0$ можно найти экспериментально, измерив выходную мощность при подключении вместо FFRM обычного волоконного зеркала с PM-волокном.


\section{Метод сканирования азимута линейной поляризации входного света}
\begin{figure}[b]
	\centering
	\includegraphics[width=1\linewidth]{figures/proposed_setup.pdf}
	\caption{Модернизированная схема тестирования FRM}
	\label{fig:proposed_setup}
\end{figure} 

Как показал наш анализ, простая схема с поляризатором не обеспечивает возможность измерения угла ротатора $\theta$ для FFRM с обычным подводящим волокном.  Причина в том, что связь регистрируемой мощности $P_r$ и угла $\theta$, описываемая (8) включает множитель $|A^2+{B^*}^2|$, меняющийся от нуля до единицы в зависимости от неконтролируемого и неизвестного сочетания коэффициентов $A$ и $B$, которые учитывают фактическую анизотропию подводящего волокна. Для преодоления данной проблемы мы предлагаем дополнить схему поворотным устройством на входе тестируемого FFRM, что показано на рисунке 2. Идея состоит в том, что вносимый поворот на угол $\alpha$ позволит дополнительно менять преобразование состояния поляризации между выходом поляризатора и зеркалом, и обеспечит достижение экстремума исходно неизвестного коэффициента, связывающего $Pr$ и $cos^22\theta$. Далее приведем анализ, обосновывающий эту идею.     
Для учета поворота введем в рассмотрение матрицу поворота:
\begin{equation}
	\label{eq:rotMatrix}
	\bm{R_\alpha} = 
	\begin{pmatrix}
		\cos\alpha & \sin\alpha \\
		-\sin\alpha & \cos\alpha
	\end{pmatrix}	
\end{equation}

При этом можно не повторять весь вывод выражения для $P_r$, а использовать уже полученное ранее выражение (8), но вместо коэффициентов матрицы подводящего волокна использовать матрицу с поворотом

\begin{equation}
	\bm{SM_\alpha} = \bm{SM}\cdot\bm{R_\alpha}= 
	\begin{pmatrix}
		A_\alpha & B_\alpha \\
		-B^*_\alpha & A^*_\alpha
	\end{pmatrix}	
\end{equation}
где
\begin{equation}
    \begin{aligned}
        A_\alpha = A\cdot\cos\alpha-B\cdot\sin\alpha \\
        B_\alpha = -A\cdot\sin\alpha+B\cos\alpha
    \end{aligned}
\end{equation}

Если в (8) заменить коэффициенты $А$ и $B$ на коэффициенты $A_\alpha$ и $B_\alpha$, то с учетом (12) после простых преобразований получим

\begin{equation}
    P_r(\alpha)=P_0\cdot\left( 1-4\Im{A\cdot B\cdot \cos\alpha-\left( \frac{B^2-A^2}{2} \right)\cdot\sin\alpha} \right)\cdot\cos^22\theta
\end{equation}

Учитывая, что комплексные коэффициенты $A_\alpha$ и $B_\alpha$ матрицы $SM_\alpha$ в общем случае подчинены только общему условию фазовой анизотропии, а именно $|A_\alpha|^2 + |B_\alpha|^2 = 1$, то как и обсуждалось для (8), в (13) множитель  между $P_0$ и $\cos2\theta$ может изменяться от 0 до 1. Но отличие состоит в том, что теперь выражение зависит от угла $\alpha$ который можно менять непосредственно вращая волокна в соединении поляризатора и FFRM. Поэтому выражение (13) записано как зависимость от $\alpha$. Рассмотрим важную составляющую полученного выражения Im{…}, которую можно представить в виде 

\begin{equation}
    \Im{A\cdot B\cdot \cos\alpha-\left( \frac{B^2-A^2}{2} \right)\cdot\sin\alpha} = a\cdot\cos\alpha -b\cdot\sin\alpha=C\cdot\cos(\alpha+\alpha_0)
\end{equation}
В (14) введены обозначения $а = \Im{A\cdot B}$ и $b = \Im{(B^2 – A^2)/2}$, а также для результирующей гармонической компоненты введены коэффициенты

\begin{equation}
    C = \sqrt{a^2+b^2} \qquad a_0 = \arctan\left( \frac{a\cdot\cos\alpha - b\cdot \sin \alpha}{a\cdot\sin\alpha + b\cdot\cos\alpha} \right)
\end{equation}
В результате выражение для регистрируемой мощности будет иметь вид  
\begin{equation}
    P_r(\alpha)=P_0\cdot\left[ 1-4 C^2\cdot\cos^2(\alpha+\alpha_0) \right]\cdot\cos^22\theta
\end{equation}
где неизвестные параметры анизотропии подводящего волокна $A$ и $B$ пересчитаны в значения коэффициентов $С$ и $\alpha_0$. Но угол является контролируемым параметром $\alpha$, который можно менять при тестировании FFRM. При этом нетрудно видеть, что, независимо от значений $С$ и $\alpha_0$, максимум зависимости от $\alpha$ будет наблюдаться при выполнении $\alpha = \alpha_0$ и определяется равенством 

\begin{equation}
    P_{rmax}=P_0\cdot\cos^22\theta
\end{equation}
Таким образом максимум регистрируемой мощности, наблюдаемый при повороте угла уже не имеет неопределенности вносимой неизвестной анизотропией подводящего волокна FFRM.

Для практической реализации измерений необходимо знать коэффициент $P_0$, который учитывает все независимые от поляризации света потери оптической мощности в тракте от источника до фотоприемника.
На практике значение этого коэффициента можно относительно просто определить в ходе калибровочного измерения, когда вместо FFRM в схему включается обычное зеркало.
Для обычного зеркала $\theta = 0$ и максимальное значение зависимости $P_r$ от угла будет соответствовать значению $P_0$.
Однако обсуждая практическую реализацию предлагаемого метода следует учесть и коэффициенты отражения зеркала.
Формально отличный от единицы коэффициент отражения $R_{FRM}$ для FFRM соответствует некоторым дополнительным потерям, учитываемым в $P0$.
Но при калибровочном измерении будет получено значение, соответствующее коэффициенту отражения $R_0$ используемого обычного зеркала.
В результате угол ротатора FFRM будет определяться выражением
\begin{equation}
    \theta =\frac{1}{2}\cdot\arccos\sqrt{\frac{P_{rmax}}{P_{r0}}\cdot\frac{R_0}{R_{FRM}}}
\end{equation}
где $P_{rmax}$ – максимум регистрируемой мощности в зависимости от угла $\alpha$, при тестировании FFRM, а $P_{r0}$ – максимум такой зависимости, полученный в ходе калибровки с обычным зеркалом.    
В случае, когда тестируемое FFRM близко к идеальному, удобно и наглядно использовать малый параметр $\delta = \theta - 45^\circ$, характеризующий поляризационную неидеальность FFRM.
Если $\theta \approx 45^\circ$, то $ \delta \ll 1$. нетрудно показать, что при выполнении приближения$ sin^22\delta \approx 4\delta^2$ из (18) следует простая оценка параметра $\delta$ по измеренным значениям $P_{rmax}$ и $P_{r0}$
\begin{equation}
    \theta =\frac{1}{2}\cdot\sqrt{\frac{P_{rmax}}{P_{r0}}\cdot\frac{R_0}{R_{FRM}}}
\end{equation}


\section{Влияние параметров элементов на результат измерений по предложенному методу.}

Приведенный в части 3 анализ теоретически обосновывает возможность относительно простых, но надежных измерений угла поворота в ротаторе FFRM.
Очевидно, что фундаментальным фактором, ограничивающим разрешающую способность таких измерений, являются флуктуации мощности, регистрируемой при измерении и в ходе калибровки.
Конкретные оценки такого ограничения зависят от конкретных используемых приборов (оптический источник и измеритель мощности).
Однако при практическом использовании данного метода следует отметить и другие факторы, которые могут вносить погрешность в результат данных измерений и ограничивать их разрешение.   
Первым характерным фактором погрешности, всегда важным при практическом использовании предложенного метода тестирования следует полагать неопределенность значений коэффициентов отражения зеркал.
В (19) они полагаются известными, однако на практике часто коэффициент отражения FFRM и FM производители задают в виде некоторого допуска, может быть дрейф от длины волны или температуры.
Для более точных измерения $\theta$ нужны дополнительные измерения для уточнения значения коэффициентов отражения.
Кроме того, под коэффициентами $R_{FRM}$ и $R_0$ в нашем рассмотрении следует понимать отношение мощности, выходящей из волоконного вывода зеркала и поступающей на вход волокна зеркала, и формально коэффициент отражения будет учитывать все потери в тракте волоконного зеркала. И если в схеме для подключения тестируемого или калибровочного зеркала используется оптоволоконный разъем, то это может внести дополнительную неопределенность. Ведь потери в разъемном соединении волокон обычно имеют существенный разброс – до нескольких десятых дБ. поэтому нужно понимать, как это отразится на результате измерений. Пусть из-за неточности информации о значениях $R_{FRM}$ и $R_0$ или значения потерь в тракте, в расчетах используется отношение $R_{FRM}/R_0$ смещение на $(1+\gamma)$, где $\gamma$ – некоторый малый параметр. Тогда измеренное значение $\delta_r$ будет отличаться от истинного значения $\delta$. Величина смещения составит    























Основным недостатком рассмотренного подхода с поляризатором на входе FFRM является чувствительность уровня регистрируемой мощности не только к углу $\theta$, к преобразованию поляризации в подводящем тракте FFRM.
Если подводящий тракт преобразует входное излучение в циркулярно поляризованное, то  выходная поляризация будет соответствовать поляризации на выходе идеального FFRM, независимо от фактического угла вращения ротатора.
В общем случае при фиксированной поляризации входного излучения, FFRM с углом вращения ротатора $\theta$ может проявлять себя как FFRM с эффективным углом вращения от $\theta$ до $45^\circ$ в зависимости от анизотропии подводящего тракта \autocite{buchilkoAnalysisStatePolarization2024}.

В случае тестирования FFRM с подводящим трактом на основе PM-волокна можно использовать модифицированные подходы, предложенные в предыдущем разделе.
Однако если подводящий тракт осуществляет произвольное преобразование поляризации, например, при использовании SM-волокна, измерения с фиксированной входной поляризацией принципиально не позволяют определить угол вращения ротатора FFRM.
В таких условиях возможным решением может быть контролируемое изменение состояния поляризации на входе FFRM.

Модернизируем схему изображенную на рис.\ref{fig:existed_setup}, добавив  между поляризатором и подводящим трактом полуволновую пластинку повернутую под углом $\alpha$ (рис.\ref{fig:proposed_setup}).
Для сохранения линейной поляризации между поляризатором и полуволновой пластинкой используется PM волокно.
Пусть как и ранее $\theta$ и $I_0$ -  угол вращения ротатора FFRM и некоторая постоянная опорная интенсивность.
Уравнение для выходной интенсивности $I_{out}$ в этом случае  может быть записано как:
\begin{equation}
	\label{eq:Iout_our}
	I_{out} =  I_0 \left( 1 - 4 Z^2 \cdot \sin^2(4\alpha - \alpha_0)\right)  \cos^2 2 \theta
\end{equation}

где
\begin{equation}
	Z = \sqrt{Im^2\left[\frac{A^2-B^2}{2}\right]+Im^2[AB]}, \qquad  \tg\alpha_0 = \frac{2Im[AB]}{Im\left[A^2 - B^2\right]}
\end{equation}

Рассмотрим (\ref{eq:Iout_our}) подробнее.
Видно, что существует угол $\alpha = \frac{\alpha_0}{4}$, при котором выходная интенсивность достигает своего максимального значения
\begin{equation}
	\label{eq:Imax}
	I_{max} = I_0\cos^22\theta
\end{equation}

Иными словами, для любого преобразования поляризации подводящим трактом можно подобрать такой угол поворота полуволновой пластинки, при котором выходная интенсивность будет достигать своего максимального значения, которое не зависит от анизотропии подводящего тракта и определяется углом фарадеевского вращения ротатора FFRM.
В соответствии с (\ref{eq:Imax}) для определения угла $\theta$ необходимо провести два измерения выходной интенсивности.
Первое с обычным зеркалом ($\theta=0$), для определения $I_0$, а второе с тестируемым FFRM.
Тогда, если $P_M$ и $P_{FFRM}$ – это соответственно интенсивности выходного излучения для обычного зеркала и FFRM, то угол вращения ротатора может быть вычислен по следующей формуле:
\begin{equation}
	\label{eq:theta}
	\theta=\frac{1}{2}\arccos\sqrt{\frac{P_{FFRM}}{P_M}}
\end{equation}

В обоих измерениях необходимо добиваться максимальной выходной интенсивности путем подбора соответствующего угла поворота полуволновой пластинки.

\section{Экспериментальная проверка предложенного метода}
\begin{figure}[b]
	\centering
	\includegraphics[width=1\linewidth]{figures/experimental_setup.pdf}
	\caption{Практическая схема тестирования FRM}
	\label{fig:experimantalScheme}
\end{figure}

Для проверки предложенного метода была собрана измерительная схема, показанная на рис.\ref{fig:experimantalScheme}.
В отличие от схемы, представленной на рис.\ref{fig:proposed_setup}, здесь отсутствует в явном виде поляризатор, однако его роль играет циркулятор, который пропускает только одну линейную поляризационную моду.
С помощью этой схемы по формуле (\ref*{eq:theta}) были определены отклонения угла фарадеевского вращения от 45° для набора FRM.
В качестве опорного зеркала для определения $P_M$ использовалось зеркало Thorlabs P5-SMF28ER-P01-1.
Результаты измерений приведены в таблице \ref{tabular:results}.
\begin{table}[h]
	\caption{Результаты измерений неидеальности зеркал Фарадея}
	\label{tabular:results}
		\begin{tabularx}{\textwidth}{|X|c|c|}
			\hline
			\thead{Зеркало} & \thead{Интенсивность выходного \\ излучения, дБм} & \thead{Неидеальность FRM, град} \\
			\hline
			\makecell{Thorlabs \\ P5-SMF28ER-P01-1} & -11.28 & - \\	
			\hline
			\makecell{AFW Technologies \\ SN: 17049033} & -47.67 & 0.43 \\
			\hline
			\makecell{AFW Technologies \\ SN: 17049037}	& -50,70 & 0.31 \\
			\hline
			\makecell{AFW Technologies \\ SN: 17049031}	& -41.98 & 0.83 \\
			\hline
			\makecell{AFW Technologies \\ SN: 17049035}	& -53.06 & 0.23 \\
			\hline
			\makecell{AFW Technologies \\ SN: 17049029} & -42.54 & 0.78 \\
			\hline
		\end{tabularx}
\end{table}

В исправных FFRM отклонение угла фарадеевского вращения обычно составляет порядка 1° \autocite{paschottaFaradayMirrors2022}, тогда как у поврежденных FFRM оно может достигать десятков градусов, вплоть до 45° в том случае, если эффект Фарадея не проявляется.
Соответственно из таблицы 1 видно, что все проверенные FFRM можно считать исправными.
Следует, однако, отметить, что при измерении не учитывались конечная экстинкция циркулятора, а также различие в коэффициентах отражения обычного зеркала и FFRM.

\section{Заключение}
В работе проанализированы представленные в научной литературе подходы к тестированию зеркал Фарадея и предложен улучшенный метод такого тестирования.
Показано, что в силу произвольности преобразования поляризации вводным трактом зеркала Фарадея, подходы использующие измерительную схему с фиксированными параметрами могут потенциально привести к некорректным результатам.
Показано, что для корректной оценки отклонения угла фарадеевского вращения в FRM достаточно использования вращающейся полуволновой пластинки как части подводящего тракта.
Приведены результаты измерения неидеальности набора FRM по предложенной методике.

\appendix
\numberwithin{equation}{section}
\section{Доказательство выражения (\ref{eq:Iout_our})}
Рассмотрим схему, представленную на рис.\ref{fig:proposed_setup}.
Пусть $\alpha$ – угол поворота полуволновой пластинки, а $\phi$ – разность фаз между собственными модами PM волокна.
Тогда матрицы Джонса для PM волокна и полуволновой пластинки могут быть записаны в следующем виде:
\begin{equation}
	PM = 
	\begin{pmatrix}
		e^{i\frac{\phi}{2}} & 0 \\
		0 & e^{-i\frac{\phi}{2}}
	\end{pmatrix},\qquad 
	HWP_\alpha = 
	\begin{pmatrix}
		\cos 2 \alpha & \sin 2 \alpha \\
		\sin 2 \alpha & -\cos 2 \alpha
	\end{pmatrix}
\end{equation}

Матрица Джонса описывающая оптическую схему правее поляризатора в отличие от (\ref{eq:M45}) будет иметь следующий вид:
\begin{equation}
	SM \cdot HWP_\alpha \cdot PM =
	\begin{pmatrix} 
		A_\alpha 	&& 	B_\alpha \\
		-B_\alpha^* && 	A_\alpha^*
	\end{pmatrix}
\end{equation}

где
\begin{equation}
	\label{eq:alpha_def}
	A_\alpha = \left( A\cos2\alpha + B\sin2\alpha\right) e^{i\frac{\phi}{2}},
	B_\alpha = \left( A\sin2\alpha - B\cos2\alpha\right) e^{-i\frac{\phi}{2}}
\end{equation}

Выражение для выходной интенсивности $I_{out}$ может быть получено из (\ref{eq:Iout_1_1}) заменой A и B на соответственно $A_\alpha$ и $B_\alpha$.

Рассмотрим отдельно множитель $\left| A^2 + {B^*}^2 \right|^2$ из выражения (\ref{eq:Iout_1_1}):
\begin{equation}
	\begin{split}
		\left| A^2 + {B^*}^2 \right|^2 &= |A|^4 + |B|^4 + 2Re\left[ (A B)^2\right]\\
		&=\left( |A|^2 + |B|^2\right)^2 - 2\left(|A B|^2 - Re\left[ (A B)^2\right]\right)\\
	\end{split}
\end{equation}	

Учитывая, что $|A|^2 + |B|^2 = 1$ и что $\forall Z \in \mathbb{C}$ выполняется равенство $Re\left[ Z^2\right] = Re^2\left[ Z\right] - Im^2\left[ Z\right]
$, приходим к следующему выражению

\begin{equation}
	\begin{split}
		\left| A^2 + {B^*}^2 \right|^2 	&=\left( |A|^2 + |B|^2\right)^2 - 4Im^2[A B] = 1 - 4Im^2[A B]
	\end{split}	
\end{equation}

Откуда получаем:
\begin{equation}
	\label{eq:Iout_our_1}
	I_{out} =  I_0 \left( 1 - 4Im^2[A_\alpha B_\alpha]\right)  \cos^2 2 \theta
\end{equation}

С учетом (\ref{eq:alpha_def}) $Im[A_\alpha B_\alpha]$ можно записать в следующем виде:
\begin{equation}
		Im[A_\alpha B_\alpha]  = Im\left[ \frac{A^2-B^2}{2} \right]\sin 4 \alpha - Im[AB]\cos 4 \alpha = Z \cdot \sin(4\alpha - \alpha_0)
\end{equation}


где
\begin{equation}
	Z = \sqrt{Im^2\left[\frac{A^2-B^2}{2}\right]+Im^2[AB]}, \qquad \tg\alpha_0 = \frac{2Im[AB]}{Im\left[A^2 - B^2\right]}
\end{equation}

Таким образом получаем окончательное выражение для интенсивности выходного излучения: 
 \begin{equation}
 	I_{out} =  I_0 \left( 1 - 4 Z^2 \cdot \sin^2(4\alpha - \alpha_0)\right)  \cos^2 2 \theta
 \end{equation}
 

\printbibliography
\end{document}
