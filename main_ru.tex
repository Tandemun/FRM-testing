\documentclass{article}
\usepackage[T2A]{fontenc}   % Кодировка шрифтов (T2A для кириллицы)
\usepackage[utf8]{inputenc} % Кодировка ввода (utf8)
\usepackage[russian]{babel}
\usepackage{tabularx}
\usepackage{amsmath}
\usepackage{amsfonts}
\usepackage{bm}
\usepackage{graphicx}%Proper images inserting
\usepackage{comment}
\usepackage{authblk}
\usepackage{makecell}
\usepackage{physics}
\renewcommand*{\Authand}{, } % Remove "and" between authors
\sloppy %To remove overfull\hbox messages from logs
\usepackage{csquotes}
\usepackage[backend=biber, style=ieee, url=false]{biblatex}
\newcommand{\frc}[2]{\raisebox{2pt}{$#1$}\big/\raisebox{-3pt}{$#2$}}    % a/b, a выше, b ниже
\addbibresource{My Library.bib}

\title{Метод тестирования исправности волоконных зеркал Фарадея}
\author[1]{Игорь Бучилко}
\author[1]{Леонид Лиокумович}
\affil[1]{Санкт-Петербургский политехнический Университет Петра Великого}

\date{Февраль 2025}

\begin{document}	
\maketitle
\begin{abstract}
В данной работе представлен новый метод тестирования волоконных зеркал Фарадея (FFRM), позволяющий точно определять угол фарадеевского вращения в условиях произвольного преобразования поляризации в подводящем тракте.
Проведен анализ существующих методов тестирования FFRM, выявлены их ключевые недостатки, связанные с зависимостью от анизотропии подводящего тракта.
Предложен модернизированный метод, основанный на использовании вращающейся полуволновой пластинки, который устраняет влияние анизотропии и обеспечивает более точное измерение угла фарадеевского вращения.
Теоретическое обоснование метода подтверждено экспериментально: проведены измерения отклонения угла фарадеевского вращения для набора FFRM, показавшие высокую точность и воспроизводимость результатов.
Предложенный метод может быть использован для тестирования FFRM в волоконно-оптических системах, включая датчики тока, интерферометры и системы квантовой криптографии.
\end{abstract}

\section{Введение}
Зеркала Фарадея (далее FRM от англ. Faraday Rotator Mirror), разработанные в 1980-х годах, стали широко применяться только последние 10 – 15 лет.
В общем случае FRM представляет собой комбинацию невзаимного ротатора Фарадея и плоского зеркала\autocite{paschottaFaradayMirrors2022}.
В исправном FRM угол фарадеевского вращения ротатора составляет 45°, в результате чего отраженное излучение сохраняет свою эллиптичность, а его азимут поворачивается на 90° относительно входного.
Волоконные зеркала Фарадея (далее FFRM, от англ. Fiber Faraday Rotator Mirror) дополнительно содержат подводящий оптический тракт, — как правило, отрезок одномодового волокна длиной несколько метров.
Существенным является то, что эффект поворота азимута на 90° сохраняется и в FFRM, несмотря на наличие подводящего тракта с произвольной фазовой анизотропией.
Это делает FFRM удобным компонентом для систем, чувствительных к изменениям поляризации.

На практике угол фарадеевского вращения может отклоняться от номинального значения из-за производственных дефектов или воздействия внешних факторов.
Механические напряжения, магнитные поля и температурные воздействия могут привести к заметному изменению угла даже у изначально корректно функционирующих устройств.
Эффекты, связанные с отклонением угла фарадеевского вращения особенно важно учитывать при разработке волоконных датчиков тока \autocite{karabulutEffectFaradayMirror2019}, \autocite{wuInfluenceImperfectFaraday2022}, систем квантовой криптографии \autocite{wangEffectImperfectFaraday2013}, \autocite{sunPassiveFaradaymirrorAttack2011}, \autocite{liangSecurityAnalysisContinuousvariable2019} и интерферометров на основе схемы Майкельсона.

Оценка угла фарадеевского вращения изолированного ротатора может быть выполнена сравнительно просто, например, с использованием схемы с двумя поляризаторами\autocite{yinFaradayAngleAccuracy2022}.
При объединении ротатора с зеркалом измерение усложняется — возникает необходимость в дополнительном разделении входного и отражённого излучения, однако сам принцип оценки сохраняется.
Существенно более сложной задача становится при наличии в составе устройства подводящего волоконного тракта, как в случае FFRM.
Преобразование поляризации в волоконном тракте носит произвольный характер и значительно затрудняет интерпретацию измерений.

Несмотря на практическую значимость, в доступной литературе проблема количественной оценки угла фарадеевского вращения в FFRM рассмотрена недостаточно подробно.
В ряде работ \autocite{wanTwoinoneFaradayRotator2014}, \autocite{sunAllFiberOpticalFaraday2010},  \autocite{sunCompactAllfiberOptical2010} предложены схемы тестирования на основе поляризаторов, однако отсутствует обоснование применяемых методик.

В данной работе проводится теоретический анализ существующих подходов к оценке угла вращения ротатора в волоконном зеркале Фарадея, выявляются их потенциальные ограничения и источники ошибок.
Кроме того, предлагается, теоретически обосновывается и экспериментально проверяется новый метод, позволяющий оценить реальный угол фарадеевского вращения в FFRM с помощью контролируемого изменения поляризации на входе тестируемого устройства.
Предложенный метод лишён недостатков существующих подходов и обеспечивает более достоверную оценку угла вращения.

\section{Анализ известных подходов}
\begin{figure}[b]
	\centering
	\includegraphics[width=1\linewidth]{figures/existed_setup.png}
	\caption{Схема тестирования с фиксированным поляризатором}
	\label{fig:existed_setup}
\end{figure}

Описанное во введении свойство FFRM поворачивать азимут поляризации на $90^\circ$ лежит в основе простой и наглядной схемы тестирования, представленной на рис.~\ref{fig:existed_setup}.
Подобная схема, в частности, применялась в работе \autocite{wanTwoinoneFaradayRotator2014}, однако её использование не сопровождалось строгим обоснованием.
Суть метода заключается в следующем.
Оптическое излучение от источника (OS) проходит через  волоконный поляризатор (P), после чего по PM-волокну поступает на вход тестируемого FFRM.
Если FFRM исправно, то поляризация отражённого излучения оказывается ортогональной входной и, таким образом, полностью подавляется поляризатором.
Соответственно мощность $P_{out}$ за поляризатором, а также измеряемая фотоприемником (OPM) мощность $P_{r}$ оказываются близки к нулю.
Если же угол фарадеевского вращения отклоняется от номинального значения $45^\circ$, то указанный режим работы нарушается и отражённое излучение может частично пройти через поляризатор.
В этом случае на фотоприемнике будет зафиксировано ненулевое значение $P_r$.

Для корректной работы схемы спектр излучения должен находиться в рабочем диапазоне FFRM.
Кроме того, для поддержания постоянной мощности $P_{in}$ на выходе поляризатора необходимо обеспечить стабильное состояние поляризации на его входе.
Поэтому, несмотря на отсутствие требований к высокой когерентности, в качестве источника излучения целесообразно использовать лазер.

Компоненты оптической схемы, предшествующие поляризатору, могут быть реализованы на базе SM волокон.
Однако участок между поляризатором и FFRM должен быть выполнен на основе PM волокна, оптическая ось которого согласована с осью поляризатора.


Рассмотрим работу этой схемы  подробнее, используя формализм матриц и векторов Джонса.
Пусть $\bm{E_{in}}$ - вектор Джонса на выходе волоконного поляризатора.
Он соответствует линейно поляризованному излучению и может быть записан в форме:
\begin{equation}
	\label{eq:Ein}
	\bm{E_{in}} = \begin{pmatrix} 1 \\ 0 \end{pmatrix}\sqrt{P_{in}} \\
\end{equation}
Здесь используется стандартная связь вектора Джонса $\bm{E}$ и мощности излучения $P$, которая с точностью до константы задается выражением \autocite{azzamEllipsometryPolarizedLight1977}: 	
\begin{equation}
	\label{eq:P-E}
	P = \bm{E^\dagger} \cdot \bm{E} = \left|\bm{E} \right|^2 
\end{equation}

Матрицы Джонса $\bm{PM}$ и $\bm{SM}$ для отрезков PM и SM волокон  могут быть записаны в следующем виде[...]:
\begin{equation}
\label{eq:PM-SM}
    \bm{PM} = 
    \begin{pmatrix}
    e^{i\frac{\phi}{2}} & 0 \\
    0 & e^{-i\frac{\phi}{2}}
    \end{pmatrix},
    \qquad 
    \bm{SM} = 
    \begin{pmatrix}
    A & B \\
    -B^* & A^*
    \end{pmatrix},
\end{equation}
Здесь $\phi$ – разность фаз между собственными модами PM волокна, а комплексные коэффициенты $A$ и $B$, удовлетворяют условию $|A|^2 + |B|^2 = 1$.
Такой вид матрицы Джонса $SM$ описывает произвольную фазовую анизотропию в отсутствии поляризационно-зависимых потерь.
Он широко используется для описания отрезков SM-волокон с относительно малыми потерями.
При этом, за исключением указанного условия, значения $A$ и $B$ могут быть произвольными, что соответствует непредсказуемой трансформации поляризации в SM-волокне.
Такая трансформация может зависеть от различных внешних факторов, включая изгибы, натяжение и температурные изменения.


Матрицы Джонса для плоского зеркала $M$, поляризатора  $P$ и невзаимного ротатора $R_\theta$ могут быть записаны в следующем виде:
\begin{equation}
	\label{eq:M-P-R}
	\bm{M} = 
	\begin{pmatrix}
		1 & 0 \\
		0 & 1
	\end{pmatrix},\qquad 		
	\bm{P} = 
	\begin{pmatrix}
		1 & 0 \\
		0 & 0
	\end{pmatrix},\qquad 
	\bm{R_\theta} = 
	\begin{pmatrix}
		\cos\theta & -\sin\theta \\
		\sin\theta & \cos\theta
	\end{pmatrix},		
\end{equation}
Следует учитывать, что конкретный вид матриц Джонса, а также правила их преобразования, при изменении направления распространении света, зависят от выбранного координатного базиса.
В настоящем анализе используется лабораторный базис, не привязанный к направлению распространения излучения.
В этом случае матрица зеркала имеет вид единичной, матрицы взаимных элементов при обратном прохождении транспонируются, а матрица невзаимного ротатора Фарадея сохраняет свой вид.

С учётом вышеизложенного, для излучения, отражённого от FFRM и прошедшего через поляризатор, вектор Джонса $\bm{E_{out}}$ запишется как:
\begin{equation}
    \label{eq:E_out}
    \bm{E_{out}} = \bm{P} \cdot (\bm{SM}\cdot \bm{PM})^T\cdot (\bm{R_\theta} \cdot \bm{M} \cdot \bm{R_\theta})\cdot (\bm{SM}\cdot \bm{PM})\cdot \bm{E_{in}}
\end{equation}	
Подставляя выражения для матриц и векторов Джонса (\ref{eq:Ein}), (\ref{eq:PM-SM}) и (\ref{eq:M-P-R}), получаем:
\begin{equation}
	\bm{E_{out}} =  e^{i\phi} \sqrt{P_{in}} \begin{pmatrix} A^2 + {B^*}^2 \\ 0 \end{pmatrix}\cos 2 \theta
\end{equation}
Тогда, в соответствии с (\ref{eq:P-E}), мощность выходного излучения $P_{out}$ будет выражаться как:
\begin{equation}
	P_{out} = P_{in} \cdot \left| A^2 + {B^*}^2 \right|^2 \cdot \cos^2 2 \theta
\end{equation}
Выражение для мощности $P_r$, регистрируемой OPM, тогда может быть записано как:
\begin{equation}
	\label{eq:Iout_1_1}
	P_{r} = P_0 \cdot \left| A^2 + {B^*}^2 \right|^2 \cdot \cos^2 2 \theta
\end{equation}
Здесь $P_0$ – опорное значение мощности учитывающее как $P_{in}$, так и поляризационно-независимые потери в схеме.
Пусть $K_{FFRM}$ — совокупный коэффициент затухания подключенного волоконного отражателя, учитывающий отличный от единицы коэффициент отражения зеркала, потери в подводящем волокне, а также на разъеме.
$K_{ms}$ — совокупный коэффициент затухания в измерительной части схемы, учитывающий потери в волокнах, на разветвителях и соединениях.
Тогда $P_0$ может быть записан как:
\begin{equation}
	\label{eq:P_0}
	P_0 = P_{in}\cdot K_{ms}\cdot K_{FFRM}
\end{equation}

Из (\ref{eq:Iout_1_1}) видно, что $P_{r}$ зависит как от угла фарадеевского вращения ротатора $\theta$, так и от параметров подводящего тракта, задаваемых коэффициентами $A$ и $B$ матрицы $SM$.
Нетрудно видеть, что  $0\le\left| A^2 + {B^*}^2 \right|\le1$.
Причем если $\left| A^2 + {B^*}^2 \right|=0$, то результат измерений совпадает с результатом для идеального FFRM, вне зависимости от значения  $\theta$.
Таким образом, неопределенность преобразования поляризации подводящим трактом FFRM, делает рассмотренную схему, в общем случае, непригодной для определения угла $\theta$.

Тем не менее, данная схема может использоваться для корректной оценки $\theta$ при условии, что подводящий тракт FFRM выполнен из PM-волокна.
При стандартной стыковке PM-волокон их оси согласованы, тогда отрезок РM-волокна явно определяется матрицей $PM$, для которой коэффициент $B$ равен нулю.
В этом случае $|A^2 + {B^*}^2| = |A|^2 = 1$, а выражение для выходной мощности принимает вид:
\begin{equation}
	\label{eq:Iout_1_2}
	P_{r} =  P_0 \cos^2 2 \theta 
\end{equation}

Таким образом $P_{r}$ определяется углом  $\theta$ и опорной мощностью $P_0$, и не зависит от преобразования поляризации в подводящем PM-волокне.
Значение $P_0$ при этом может быть определено с помощью дополнительного калибровочных измерения.
Отметим, что к такому же выводу можно придти, если в качестве подводящего волокна используется не PM, а SPUN волокно, сохраняющее круговую поляризацию.


\section{Метод сканирования азимута линейной поляризации входного света}
\begin{figure}[b]
	\centering
	\includegraphics[width=1\linewidth]{figures/proposed_setup.png}
	\caption{Модернизированная схема тестирования FRM}
	\label{fig:proposed_setup}
\end{figure} 
Как показано в предыдущем разделе, простая схема с поляризатором не позволяет оценить угол вращения ротатора $\theta$ для FFRM с подводящим трактом на основе SM-волокна.
Это связано с тем, что зависимость регистрируемой мощности $P_r$ от угла $\theta$, описываемая (\ref{eq:Iout_1_1}) содержит неконтролируемые и неизвестные коэффициенты $A$ и $B$, характеризующие анизотропию подводящего волокна.

Возможным решением этой проблемы является введение дополнительного контролируемого воздействия на состояние поляризации на входе FFRM.
В простейшем случае это может быть вращение азимута поляризации, с помощью поворотного устройства $R$, как показано на рис.\ref{fig:proposed_setup}.
В результате регистрируемая мощность $P_r$ становится функцией угла поворота $\alpha$.
Изменяя этот угол можно подобрать такое состояние поляризации на входе FFRM, при котором $P_r$ достигает максимального значения, не зависящего от коэффициентов $A$ и $B$.

Поворот азимута поляризации может быть описан матрицей поворота $R_\alpha$
\begin{equation}
	\label{eq:rotMatrix}
	\bm{R_\alpha} = 
	\begin{pmatrix}
		\cos\alpha & \sin\alpha \\
		-\sin\alpha & \cos\alpha
	\end{pmatrix}	
\end{equation} 
Такой поворот может быть интерпретирован как составляющая анизотропии подводящего волокна.
Это позволяет для определения $P_r$ использовать уже полученное выражение (\ref{eq:Iout_1_1}), которое с помощью несложных преобразований можно представить в следующем виде: 
\begin{equation}
    \label{eq:1-4Im}
    P_{r} = P_0 \cdot \left(1-4\Im^2[AB] \right) \cdot \cos^2 2 \theta
\end{equation}
С учетом вносимого поворота поляризации, матрица подводящего волокна может быть записана как:
\begin{equation}
    \bm{SM_\alpha} = \bm{SM}\cdot\bm{R_\alpha}= 
    \begin{pmatrix}
		A_\alpha & B_\alpha \\
		-B^*_\alpha & A^*_\alpha
    \end{pmatrix}	
\end{equation}
где
\begin{equation}
   \label{eq:Aa-Ba}
    \begin{aligned}
        A_\alpha = A\cdot\cos\alpha - B\cdot\sin\alpha \\
        B_\alpha = A\cdot\sin\alpha + B\cdot\cos\alpha
    \end{aligned}
\end{equation}
Заменяя в (\ref{eq:1-4Im}) $A$ и $B$ на $A_\alpha$ и $B_\alpha$ получаем следующее выражение для выходной мощности:
\begin{equation}
    \label{eq:Pr_our}
    P_r(\alpha)=P_0\cdot\left( 1-4Im^2\left[A B\cdot \cos2\alpha -\left( \frac{B^2-A^2}{2} \right)\cdot\sin2\alpha\right] \right)\cdot\cos^22\theta
\end{equation}
Нетрудно видеть, что компоненту $\Im[\ldots]$ можно представить в виде гармонической функции:
\begin{equation}
    \Im\left[A B\cdot \cos2\alpha -\left( \frac{B^2-A^2}{2} \right)\cdot\sin2\alpha\right] = C\cdot\sin(2\alpha-\alpha_0)
\end{equation}
где
\begin{equation}
    C^2=\Im^2[AB]+\Im^2\left[\frac{B^2-A^2}{2}\right] \qquad \tan\alpha_0=\frac{2\Im[AB]}{\Im[B^2-A^2]}
\end{equation}
Выражение (\ref{eq:Pr_our}) таким образом может быть записано в следующей форме:
\begin{equation}
    \label{eq:Pr_our_1}
    P_r(\alpha)=P_0\cdot\left[1-4C^2\sin^2(2\alpha-\alpha_0)\right]\cdot\cos^22\theta
\end{equation}
При $\alpha=\alpha_0/2$ зависимость $P_r(\alpha)$ достигает максимума равного:
\begin{equation}
    \label{eq:P_r_max}
    P_r^{max}=P_0\cdot\cos^22\theta
\end{equation} 
Хотя угол $\alpha$, при котором достигается максимум $P_r(\alpha)$, определяется параметрами анизотропии подводящего волокна, само максимальное значение $P_r^{\max}$ от них не зависит.
Это позволяет использовать $P_r^{\max}$ для оценки параметров непосредственно FRM.

На практике, как правило, важна не величина $\theta$, а её отклонение от номинального значения $45^\circ$, характеризующее поляризационную неидеальность FFRM.
Обозначим это отклонение как $\delta =| \theta - 45^\circ|$.
Проводя в (\ref{eq:P_r_max}) замену $\theta = 45^\circ \pm \delta$ получаем:
\begin{equation}
    \label{eq:P_r_max_1}
    P_r^{max}=P_0\cdot\sin^22\delta
\end{equation} 

Как уже говорилось, проблема неизвестного значения $P_0$ может быть решена с помощью дополнительной калибровки. Для этого можно предложить измерение, при котором вместо FFRM в схему устанавливается обычное волоконное зеркало (FM).
Такое зеркало может выполнит роль калибровочного в том смысле, что для него выполняется $\delta = 45^\circ$. Поэтому значение $P_r^{max}$, в этом случае будет равно $P_0$. Однако следует учесть, что выражение (\ref{eq:P_0}) в этом случае должно содержать коэффициент потерь в калибровочном зеркале $K_{FM}$ вместо $K_{FFRM}$.  Обозначим $P_{FFRM}^{max}$ – максимум регистрируемой мощности в зависимости от угла $\alpha$, при тестировании FFRM, и $P_{FM}^{max}$ – максимальная мощность, зарегистрированная при калибровке с обычным зеркалом. Тогда полагая известными затухания в зеркалах $K_{FM}$ и $K_{FFRM}$, с учетом (\ref{eq:P_0}) и (\ref{eq:P_r_max}) получим выражение для оценки $\delta$:
\begin{equation}
    \label{eq:delta}
    \delta =\frac{1}{2}\cdot\arcsin\sqrt{\frac{P_{FFRM}^{max}\cdot K_{FM}}{P_{FM}^{max}\cdot K_{FFRM}}}
\end{equation}
В случае, когда тестируемое FFRM близко к идеальному, то есть когда $\delta \ll 1$, предыдущее выражение может быть сведено к:
\begin{equation}
    \label{eq:delta_est}
    \delta =\frac{1}{2}\sqrt{\frac{P_{FFRM}^{max}\cdot K_{FM}}{P_{FM}^{max}\cdot K_{FFRM}}}
\end{equation}

\section{Факторы ограничения разрешающей способности измерений по предложенному методу.}

Приведенный в части 3 анализ теоретически обосновывает возможность относительно простых, но надежных измерений отклонения угла в ротаторе FFRM от "идеального" значения.
Анализ метрологических характеристик данного метода не входит в рамки данной статьи, но здесь можно дать некоторые пояснения 
о возможных факторах ограничения возможности регистрировать отклонение $\delta $ от нуля. Очевидно, что фундаментальный фактор, ограничивающий такую возможность это минимальное значение мощности, которую может измерить OPM. В сочетании с мощностью OS и потерями в тракте прохождения света это ограничит минимальный обнаруживаемый уровень $\delta $.
Однако при практическом использовании данного метода следует отметить и другие факторы неопределенности, которые могут искажать результат данных измерений и ограничивать их разрешение.   
Характерным фактором возможной погрешности, присутствующим при практическом использовании предложенного метода тестирования, является неопределенность в значений коэффициентов отражения зеркал. В (\ref{eq:delta_est}) они полагаются известными, однако на практике часто коэффициенты отражения волоконных зеркал (в том числе фарадеевских) производители задают в виде некоторого допуска, может быть дрейф от длины волны или температуры. Более того, при рассматриваемых измерениях удобно подключать тестируемое зеркало через оптовлоконный разъем, Поэтому фактические значениях $K_{FFRM}$ и $K_{FM}$ будут иметь значительную неопредленность (на уровне 0,5 дБ), которую нельзя найти дополнительными измерениями. Для пояснения влияния данного фактора, примем, что используемое при расчете отношение $K_{FRM}/K_{FM}$ отличается от истинного на коэффициент $(1 + \gamma)$, где $\gamma$ – некоторый малый параметр. Нетрудно видеть, что в результате измеренная величина будет задавиться выражением   
\begin{equation}
    \label{eq:delta_r(gamma)}
    \delta_r =\frac{1}{2}\sqrt{\frac{P_{FFRM}^{max}\cdot K_{FM}}{P_M^{max}\cdot K_{FFRM}}\cdot(1+\gamma)}\approx \delta\left(1+\frac{\gamma}{2}\right)
\end{equation}
где в приближенной части отброшены компоненты второго порядка малости по $\gamma$.
Как видно из (\ref{eq:delta_r(gamma)}) неточное знание фактического значения отношения $K_{FRM}/K_{FM}$ дает мультипликативное смещение результата измерений. Таким образом некотрая неопределеность в значениях $K_{FFRM}$ и $K_{FM}$ обычно имеющая место для реальных элементов не ограничивает возможность определения малых значений  $\delta $.  

Другим характерным неустранимым на практике фактором погрешности для предложенного метода измерений является конечная экстинкция поляризатора, поскольку реальный поляризатор обладает конечным относительным пропусканием $t$ для поляризации ортогональной оси поляризатора. Полноценный анализ этого фактора непрост, поскольку необходимо рассматривать состояние поляризации света поступающего к поляризатору от источника. Однако для иллюстрации возможного влияния данного фактора рассмотрим упрощенный случай, предполагая, что излучение $\bm{E_{in}}$ модно считать горизонтально поляризованным. Тогда неидеальность поляризатора проявится только при прохождении света от зеркала в обратном направлении, что относительно легко учесть в теоретической модели.  Матрицу поляризатора с конечной экстинкцией $t$ можно записать в виде:
\begin{equation}
	\bm{P_{real}} = 
	\begin{pmatrix}
		1 & 0 \\
		0 & \sqrt{t}
	\end{pmatrix}	
\end{equation}

Подставляя $\bm{P}_{\text{real}}$ вместо $\bm{P}$ в произведение (\ref{eq:E_out}), по аналогии с (\ref{eq:Pr_our_1}) получаем выражение для мощности выходного излучения:
\begin{equation}
    P_r(\alpha)=P_0\cdot\{(1-t)\left[1-4C^2\sin^2(2\alpha-\alpha_0)\right]\cdot \cos^22\theta+t \}
\end{equation}

Зависимость $P_r(\alpha)$, как и ранее, достигает максимума при $\alpha=\alpha_0/2$: 
\begin{equation}
    \label{eq:Pr_max(delta,t)}
    P_r^{max}=P_0\cdot\left[(1-t)\cdot \sin^22\delta+t \right]
\end{equation}
Если для простоты пренебречь отличием отношения $K_{FRM}/K_{FM}$ от единицы, то с учетом предыдущего выражения получим оценку измеренной величины угла ротатора:
\begin{equation}
    \label{eq:delta_est(delta, t)}
    \delta_r=\frac{1}{2}\cdot\arcsin\sqrt{(1-t)\cdot \sin^22\delta+t}
\end{equation}
Из этого выражения следует, что влияние конечной экстинкции поляризатора дает как мультипликативное, так и аддитивное смещение измеренной величины. В частности, если выполняется условие $4\delta^2\gg t$, то существенной будет мультипликативная погрешность, когда
\begin{equation}
    \delta_r \approx \delta\left(1-\frac{t}{2}\right)
\end{equation}
т.е. погрешность по структуре аналогична структуре (\ref{eq:delta_r(gamma)}). Однако если выполняется обратное условие $4\delta^2\ll t$, то получим:
\begin{equation}
    \delta_r \approx \frac{\sqrt{t}}{2}
\end{equation}
т.е. когда зеркало приближается по поляризационным свойствам к идеальному, независимо от значения  $\delta $ измеренное значение будет ограничено конечным значением $t$. Этот пример соответствует упрощенному рассмотрению влияния конечного значения экстинкции поляризатора только при обратном распространении света. Однако он показывает, что этот фактор может ограничить разрешающую способность измерения отклонения $\delta $ от нуля.   


\section{Экспериментальная проверка предложенного метода}
\begin{figure}[b]
	\centering
	\includegraphics[width=1\linewidth]{figures/experimental_setup.pdf}
	\caption{Практическая схема тестирования FRM}
	\label{fig:experimantalScheme}
\end{figure}

\begin{figure}[b]
	\centering
	\includegraphics[width=1\linewidth]{figures/fm_curves.png}
	\caption{Зависимость $P_r(\alpha_{HWP})$ при разных положениях лепестков PC}
	\label{fig:fm_curves}
\end{figure}

\begin{figure}[b]
	\centering
	\includegraphics[width=1\linewidth]{figures/magnet.png}
	\caption{Положение FFRM вблизи магнита}
	\label{fig:magnet}
\end{figure}

Для проверки предложенного метода была собрана измерительная схема, показанная на рис.\ref{fig:experimantalScheme}.
Вместо комбинации разветвителя с поляризатором использовался трёхпортовый циркулятор, пропускающий только одну поляризационную моду.
Такое решение соответствует предлагаемому принципу тестирования FFRM, но является более удобным с практической точки зрения.
В качестве поворотного устройства для управления поляризацией на входе FFRM применялась вращающаяся полуволновая пластинка.
В случае линейно поляризованного входного излучения, поворот пластинки на угол $\alpha_{HWP}$ приводит к повороту азимута на угол $2\alpha_{HWP}$. 
Для имитации различных вариантов преобразования поляризации подводящим трактом FFRM использовался 3-paddle polarization controller (PC).
Хотя он не требуется для выполнения измерений, его использование позволяет наглядно продемонстрировать влияние анизотропии подводящего волокна зеркала на характер зависимости измеряемой мощности от поворота пластинки.
В частности на рис.\ref{fig:fm_curves} приведены нормированные зависимости $P_r(\alpha_{HWP})$ для обычного волоконного зеркала при различных положениях лепестков контроллера.
Полученные зависимости хорошо согласуются со структурой выражения (\ref{eq:Pr_our}), что косвенно подтверждает адекватность представленного теоретического описания.


Для апробации предложенного метода тестирования были проведены измерения пяти образцов FFRM.  
Для каждого зеркала фиксировалось максимальное значение мощности $P_r^{max}$, достигаемое при вращении полуволновой пластинки.
Для каждого FFRM измерения проводились при различных положениях PC, а затем усреднялись.
При этом вариация $P_r^{max}$ для каждого зеркала составила менее 0,05 дБм, что также говорит о справедливости выражения (\ref{eq:Pr_our}).
Отклонение  $\delta$ для каждого образца FFRM рассчитывалось в соответствии с (\ref*{eq:theta}).
Результаты измерений приведены в таблице \ref{tabular:results}.
\begin{table}[h]
	\caption{Результаты измерений неидеальности зеркал Фарадея}
	\label{tabular:results}
		\begin{tabularx}{\textwidth}{|X|c|c|}
			\hline
			\thead{Зеркало} & \thead{Интенсивность выходного \\ излучения, дБм} & \thead{Неидеальность FRM, град} \\
			\hline
			\makecell{Thorlabs \\ P5-SMF28ER-P01-1} & -11.28 & - \\	
			\hline
			\makecell{AFW Technologies \\ SN: 17049033} & -47.67 & 0.43 \\
			\hline
			\makecell{AFW Technologies \\ SN: 17049037}	& -50,70 & 0.31 \\
			\hline
			\makecell{AFW Technologies \\ SN: 17049031}	& -41.98 & 0.83 \\
			\hline
			\makecell{AFW Technologies \\ SN: 17049035}	& -53.06 & 0.23 \\
			\hline
			\makecell{AFW Technologies \\ SN: 17049029} & -42.54 & 0.78 \\
			\hline
		\end{tabularx}
\end{table}

Для всех образцов FFRM значение $\delta$ составило менее 1°, что соответствует паспортным характеристикам зеркал.
Следует, однако, отметить, что при измерении не учитывались конечная экстинкция циркулятора, а также различие в коэффициентах отражения обычного зеркала и FFRM.

Для демонстрации чувствительности предлагаемого метода была смоделирована ситуация с «неисправным» зеркалом. Образец 1 помещался в постоянное магнитное поле, как показано на рис.\ref{fig:magnet}. Использовались два варианта размещения: край корпуса зеркала вводился внутрь магнитного зазора на 5 мм и на 16 мм. Индукция поля в рабочей области составляла около 250 мТл.

Результаты измерений приведены в таблице 2.

\begin{table}[h]
	\caption{Результаты измерений неидеальности зеркал Фарадея}
	\label{tabular:results_1}
		\begin{tabularx}{\textwidth}{|X|c|c|}
			\hline
			\thead{Configuration}        & \thead{$P_r^{max}$, dBm} & \thead{Неидеальность FRM, град} \\
			\hline
			\makecell{Without magnet}    & -50.70 & 0.31 \\
			\hline
			\makecell{First position}    & -17.19 & 15.08 \\
			\hline
			\makecell{Second position}   & -13.64 & 24.63 \\
			\hline
			\makecell{After removing}    & -49.23 & 0.36 \\
			\hline
		\end{tabularx}
\end{table}

Внешнее магнитное поле увеличивало отклонение $\delta$ до 15° и 25° в зависимости от положения.
После удаления магнита угол $\delta$ возвращался к исходному значению, что подтверждает обратимость эффекта.
В отличие от измерений по таблице 1, которые можно сопоставить с паспортными данными, значения $\delta$, полученные в магнитном поле, не имеют известных эталонов, но хорошо иллюстрируют эффективность метода при диагностике «неисправных» зеркал.


\clearpage
\section{Заключение}
В работе проанализированы представленные в научной литературе подходы к тестированию зеркал Фарадея и предложен улучшенный метод такого тестирования.
Показано, что в силу произвольности преобразования поляризации вводным трактом зеркала Фарадея, подходы использующие измерительную схему с фиксированными параметрами могут потенциально привести к некорректным результатам.
Показано, что для корректной оценки отклонения угла фарадеевского вращения в FRM достаточно использования вращающейся полуволновой пластинки как части подводящего тракта.
Приведены результаты измерения неидеальности набора FRM по предложенной методике.

\printbibliography
\end{document}
